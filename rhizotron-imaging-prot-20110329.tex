\documentclass[11pt]{article}
\usepackage{graphicx}
\usepackage{multirow}
\usepackage[tiny,compact]{titlesec}

\title{Minirhizotron Image Collection at the Energy Farm}
\author{Chris Black, black11@igb.uiuc.edu\\503 929-9421}
\date{Rev. 2, 24 May 2011}

\begin{document}
\maketitle


\section{Equipment}
	\begin{itemize}
		\item{Rhizotron camera, with case}
		\item{I-Cap laptop}
		\item{Batteries (at least 2 for a full day)}	
		\item{Step stool}
		\item{Swab stick}
		\item{Blue paper towels}
		\item{Flashlight}
		\item{Plot map with tube numbers}
		\item{Water sensor ({\it Optional, and not a replacement for the flashlight!})}
	\end{itemize}

	
\section{[mis-] Labeling} 
	There are 96 tubes, with four in each of the small plots (blocks 1-4) and 8 in each of the large plots (block 0). Originally, the tubes were labeled by block, crop, and collar (e.g. `3S2'), but iCap thinks in tube numbers (e.g. `66'), so we're re-labeling with tube number as we replace tubes. Tubes that have been replaced since 2009 are now labeled with their tube number, while original tubes are still labeled with the block/crop/collar code. The camera case is `tube 97', to be used for calibration images. Tube 98 used to be tube 36; we installed a new tube but then decided the old one was still okay. Collect images from both of them.
	
	There is at least one tube that is labeled incorrectly: 0S11 is labeled in black as 0S12 (True 0S12 is labeled in blue). If you find any others that are mislabeled or missing their labels, refer to the map and convince yourself what the correct number is, save the images with that number, and save a note at location 1 explaining the mix-up.

\begin{table}[htbp]
	\begin{center}
		\caption{\label{conversion}Tube numbers in each block.}
		\vskip1em
		\begin{tabular}{rll|rll}
Block & Crop & Tubes & Block & Crop & Tubes\\
\hline
0	& Corn 			& 1--8 			& 3	& Corn 			& 17--20\\ 	
	& Miscanthus 	& 25--32		& 	& Miscanthus 	& 41--44\\
	& Switchgrass 	& 49--56		& 	& Switchgrass 	& 65--68\\
	& Prairie 		& 73--80		&	& Prairie 		& 89--92\\
\hline								
1	& Corn 			& 9--12			& 4	& Corn 			& 21--24\\	
	& Miscanthus 	& 33--36, 98	&	& Miscanthus 	& 45--48\\
 	& Switchgrass 	& 57--60		& 	& Switchgrass 	& 69--72\\
	& Prairie 		& 81--84		& 	& Prairie 		& 93--96\\
\hline
2	& Corn 			& 13--16		&	\multicolumn{2}{r}{Case (for calibration)}& 97\\
 	& Miscanthus 	& 37--40		& 	&				&\\
	& Switchgrass 	& 61--64		& 	&				&\\
	& Prairie 		& 85--88		& 	&				&\\
		\end{tabular}
	\end{center}
\end{table}

\section{Experiment setup}
	Open I-Cap and press ``Open an existing database." Select `EF2011(EHD).btc' and press ``Open''. If it opens without a hitch, enter your initials and check that the gather direction is DOWN, then move on to image capture.
	If I-Cap crashes with a `subscript out of range' message, set up a new experiment and we'll merge the images back into EF2011 later.
\begin{enumerate}
	\item{Select ``Create a new experiment database.''}
	\item{Give the experiment a name something like ``EF2011AUG29''.}
	\item{Principal Investigator: EHD. Total Number of Tubes: 98. Image Spacing: 13.5. Tube Length: 1620 mm.}
	\item{Image Compression: JPG. Image Captioning: Off.}
	\item{Ignore the proposed start and end dates.}
	\item{Press ``Next''}
	\item{Leave the Quick Notes fields blank.}
	\item{Enter your initials and set the gather direction to DOWN.}
	\item{Save a calibration image.}
	\item{Proceed with image capture.}
\end{enumerate}

		
\section{Calibration}
	The camera's widest zoom setting leaves excessive vignetting, so it's necessary to zoom in slightly, but the zoom button doesn't give a reproducibly exact zoom level. Thus, calibrate. 
	
\begin{enumerate}
	\item{Tell I-Cap to go to tube 97. Go to the first location without an image (e.g. location 2 if you saved one image yesterday, location 3 when you decide to recalibrate after lunch, etc.)}
	\item{Use the ``Notes'' feature, \textit{before} capturing an image, to record enough detail that it's clear which tubes this calibration is good for (Usually this will just be something like ``calibration for 2011-08-29'', but add detail as needed, e.g.  ``Zoom switch knocked when detaching from tube 23; this calibration is for images taken after 1:30 PM on 2011-08-30.''}
	\item{Put the camera into its carrying case and rotate it until you can see the 14x18 mm grid of lines spaced 1 mm apart.}
	\item{Focus until the ink spatters around the lines show crisply.}
	\item{Adjust the zoom until the image area is as close as possible to 12x17 mm. (Realistically, that means somewhere between 16 and 18 mm horizontal. If the corners of the image are way darker than the center, zoom in more.}
	\item{Save the image.}
	\item{Move the camera to the correct tube number and take images. Do not disturb the zoom setting. Adjust the focus setting as needed, but it should be okay as it is.} 
\end{enumerate}
	
	One calibration per day should be enough, \textit{if} you're certain that you haven't touched the zoom button since calibrating. More calibrations, if they're clearly labeled, won't hurt anything (other than your speed and patience). If in doubt, take another one.


\section{Tube Swabbing}
	The main purpose of this is to remove water in the tube, but you should also  swab if there seems to be any other crud affecting visibility on the inside of the tube.
	\begin{enumerate}
		\item{Shine a flashlight down the tube. If everything looks clean and dry, skip swabbing and move on to image collection. If you can't tell whether it's clean and dry, get a better flashlight.}
		\item{If swabbing is needed, mount a towel onto the swab stick and extend the handle all the way. Make sure the towel is clean; grit and dirt will scratch the inside of the tube.}
		\item{Run the swab down the tube, rotating it as you push. Pull it back out, rotating it as you lift.}
		\item{If the swab come out  wet or dirty, repeat until it comes out dry and clean. This can be slow and gross, but don't give up. I have emptied entirely-full tubes using nothing but lots of towels wrung out over and over again.}
		\item{Shine the flashlight and look down the tube again. Don't insert the camera until you've convinced that the whole tube is safe for electronics. \textit{Never put the camera down a tube if you can't see to the bottom of it!}}
		\item{If you've just removed a whole lot of water, wait a few minutes (maybe go collect images from another tube and come back) and re-recheck it to make sure the tube isn't refilling immediately. Tell me about these tubes as soon as possible so I can fix them.}
\end{enumerate}
	
\section{Image Capture}
	Bear in mind that your main job during image collection is to make sure that the computer is always told accurately where it is. It has no way of double-checking you, and an image saved in the wrong place is worse than useless.
	
	There are innumerable quirks and bugs in the I-Cap system, so this section is intended as a reminder for the previously trained rather than as a step-by-step protocol.
	
\subsection{At Every Tube}
	\begin{enumerate}
		\item{Set the tube number correctly.}
		\item{Adjust the camera until you can see the label written on the tube (unlock the rotation pin and twist if you have to). Go to the location that most closely corresponds to that index location (\textit{should} always be location 1 if I was less sloppy with my label placement).}
		\item{Convince yourself that the label is correct for the tube you think you're imaging. Look at the tube map again. Be suspicious.}
		\item{Make sure the image is in focus and well-lit.}
		\item{If you have notes to save about the whole tube (say, ``swabbed out 1 inch of water'' or ``Chris, you bonehead, this tube is labeled with the wrong number''), attach them to the same image that shows the tube label.}
		\item{Save the image showing the label.}
		\item{Go to Location 5.}
		\item{Set up Autorun with a 5-location increment and not going past Location 110.}
		\item{Take a deep breath, get your hands ready, and hit control-U. Keep the camera in sync while it saves images for the remainder of the tube. If the autorun outruns you and saves blurry images, go back and correct them.}
	\end{enumerate}	

\subsection{Some General Reminders}

	When dismissing the Go window, avoid an unwanted image save by using Control-O, not the enter key.
	
	If anything seems wonky, save a note explaining it.
	
	Remember that notes are associated with images, not locations! If you make a note and then don't save an image at that location, the note isn't saved either.
	
	When using Go, beware the location-doesn't-update bug. It's okay to save an image at the new location when it's still displaying the old one, \textit{if} you're religiously sure you entered the new location correctly.
	
	If there are multiple images for a location and no notes to help me out, I will assume that the one captured later is correct and the one captured earlier is suspect. It's supposed to be quite difficult to accidentally capture a second image at an already-imaged location, but there seems to be a bug that lets it happen sometimes if you're not paying attention. Try to minimize this by working systematically and only setting the location back to an already-imaged one if it's necessary to correct a previous error.


\section{Equipment care}
	We're trying to haul a computer and a bunch of delicate optics around a dusty, pollen-infested field. Everything involved is hideously expensive. Scared? Me too. In general, treat everything as gently as you can and keep it as clean as it's practical to do.
	
	Avoid kinks and jerks on the camera cable. Disconnect it (turn off the case power first!) whenever you move the laptop.
	
	Don't put the camera down a tube that has water in it. It's less waterproof than they advertise. 
	
	If the camera head does get wet, turn it off immediately. Dry the outside with paper towels. Clean the lens area with alcohol and Q-tips. Clean the top ends of the light bulbs with alcohol, then wrap a bit of masking tape around them to get enough of a grip to pull them (gently) out of their sockets. Dry the insides of the sockets with the corners of a paper towel. Let everything air-dry thoroughly. \textbf{Do not turn the camera back on!} Bring it back to the lab and I'll take care of it.
	
	
If you have any trouble or something's not clear, give me a call (503 929-9421). 
\end{document}
	